\section{Sp�cification}
\label{sec:specification}

\paragraph{Avant de commencer} Avant toute action sur le logiciel, l'utilisateur doit cr�er un nouveau projet.

\subsection{Dessiner son UML}
\label{sec:spec1}

\paragraph{} L'utilisateur peut, si il le souhaite, dessiner un diagramme UML. Pour cela, plusieurs boutons lui sont propos�s. Il peut cr�er diff�rents packages pour ranger ses classes, il peut �galement cr�er des classes, classes abstraites et interfaces qu'il pourra, � leur cr�ation ou plus tard, incorporer dans le package de son choix. Dans le moteur du logiciel, il faut savoir que des donn�es sont tout simplement cr��es et stock�es.
Initialement, elles sont plac�es dans le package par d�faut. Une fois ses classes, classes abstraites et interfaces cr��es, il pourra les associer entre elles comme par exemple lier deux classes avec un h�ritage ou encore une composition. Il peut �galement, si il le souhaite, impl�menter une classe avec une interface.\\Si il se trompe, l'utilisateur peut supprimer un lien mais �galement �diter voir supprimer une classe. Il n'aura pas besoin de supprimer les packages, seuls les packages ayant au moins une classe seront affich�s.


\subsection{Convertir son UML en code Java}
\label{sec:spec2}

\paragraph{} Lors de la conversion de l'UML en code Java, les donn�es qui ont �t� cr��es lors de la cr�ation des diff�rents packages, classes ... sont "lu" a partir de l'objet "Project" qui contient toutes ces donn�es. Par la suite, il suffit de r�cup�rer, dans ces donn�es, le code correspondant � leur repr�sentation en code Java.

\subsection{Ecrire un code Java}
\label{sec:spec3}

\paragraph{} Une fois son projet converti en code, l'utilisateur peut commencer un d�but de code simple sur ces donn�es. On parle ici de code simple car il ne pourra cr�er directement ni package ni classe � travers l'�diteur de fichier incorporer dans le logiciel.

\subsection{Convertir un code Java en diagramme UML}
\label{sec:spec4}

\paragraph{} A travers ce dernier outils, l'utilisateur � la possibilit� d'effectuer une conversion inverse, il lui suffit de s�lectionner le dossier qui contient les packages et les fichiers .java de son projet pour que le moteur du logiciel lise ce code et le d�compose afin d'en extraire les donn�es et les repr�senter sous forme d'UML.