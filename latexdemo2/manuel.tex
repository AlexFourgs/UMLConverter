\section{Manuel Utilisateur}
\label{sec:manuel}

\paragraph{D�s l'ouverture} l'utilisateur doit aller dans "File" et s�lectionner "New Projet".

\subsection{Comment dessiner son UML}
\label{sec:man1}

\noreffig{images/vue_uml.png}{16.00cm}{9.0cm} \\

\paragraph{} L'utilisateur peut cr�er des classes, classes abstraites et interfaces � l'aide des boutons pr�sents sur le c�t� dans la partie "Package, classes \& inc". Il peut �galement cr�er � l'aide du bouton "Package" un ou plusieurs packages qui serviront � ranger ses classes cr��es pr�c�demment ou par la suite. Dans la cas o� les classes sont cr��es avant les packages, l'utilisateur dispose d'un menu sous son UML qui lui permettra de supprimer ou d'�diter ses classes et de les placer, � l'aide d'un menu d�roulant, dans les packages existants. Si il cr�� ses classes apr�s ses packages, il pourra les ranger dans les packages voulus lors de la cr�ation � l'aide d'un menu d�roulant. Le menu sous son UML sera �galement pr�sent si il souhaite se corriger.

\subsection{Convertir son UML en code Java ou l'exporter en image}
\label{sec:man2}

\paragraph{} Apr�s avoir fini son UML, l'utilisateur peut exporter son UML en image ou en convertir en code Java. Pour cela, il lui suffit de s�lectionner son choix dans le menu "File".

\newpage

\subsection{Convertir un code Java en diagramme UML}
\label{sec:man3}

\noreffig{images/vue_code.png}{16.00cm}{9.0cm} \\

\paragraph{} Si l'utilisateur souhaite faire l'op�ration inverse � celle pr�vue de base, c'est-�-dire g�n�rer un diagramme UML � partir d'un code Java, il doit sauvegarder son code, ouvrir une nouvelle fois le logiciel, cr�er un nouveau projet et ensuite choisir l'option ".java to UML" dans le menu "File".


