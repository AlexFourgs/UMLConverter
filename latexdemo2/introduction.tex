\newpage
\section{Introduction}
\label{sec:introduction}


\subsection{Cadre}
\label{sec:intro1}

\paragraph{} Dans le cadre du cours de G�nie Logiciel durant notre seconde ann�e de licence d'informatique, nous avons �t� amen�s � r�aliser un projet de programmation en Java. Etant en bin�me, nous avons eu le choix entre 5 projets, nous avons choisis le sujet UML.

\subsection{Pourquoi ce sujet ?}
\label{sec:intro2}

\paragraph{Choix} Pourquoi ce sujet en particulier ?

\begin{itemize}
\item Pour son utilit�.
\item Un sujet qui concerne un outil que l'on utilise.
\end{itemize}

\paragraph{} Un UML est un outil puissant nous permettant de concevoir et de mieux comprendre le fonctionnement d'un logiciel. Lorsque nous avons un projet en t�te, durant la phase de conception nous sommes souvent amen�s � passer du temps sur l'UML de notre projet. Ceci dit, le temps que nous passons sur l'UML est du temps en moins sur la r�alisation.
\paragraph{} A travers ce projet, nous avons vu un outils permettant de concevoir un logiciel tout en le r�alisant, car vous n'aurez pas � prendre du temps � retranscrire votre UML en langage de programmation, cet outil le fera pour vous. Tout l'enjeu de ce logiciel se joue ici, le temps et la facilit� � transformer votre id�e en un d�but de logiciel, et c'est particuli�rement pour cela que nous avons choisis ce sujet.
\paragraph{} De plus, au vue de notre situation d'�tudiant en informatique et d�couvrant le monde de la programmation orient�e objet depuis peu, nous avons eut l'occasion de faire des UML et de travailler dessus. Cette seconde raison nous a aussi pouss� � choisir ce sujet.\\En effet, nous pensons nous servir de ce logiciel dans le futur.
